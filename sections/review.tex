\documentclass[main.tex]{subfiles}
\begin{document}
  
  \section{Requirements and Issues}
    
    \subsection{Requirement Review}
      
      It should likely be argued that the software requirements were not very detailed, due to lack of direct contact with the customer. As such all non-functional requirements were satisfied. This did place some constraints on the software, but it is at this point not very foreseeable just how limiting these decisions will be. Implementation thus far has not been very mathematically challenging. Luckily a powerful Linear Algebra Library is available in Java 11 \cite{} that would continue to satisfy functional requirements if it becomes necessary.
      \\\\
      The core requirement has been fulfilled, to produce a function classifier based on calculus. However, at this time no further data is extracted. Very likely, this will make use of Regression, and largely be a copy of the involved mathematics.
      \\\\
      As suggested in the project requirements, thus far the software is a prototype that proves the concept and is build to be improved upon.
      
    \subsection{Known Issues}
      
      A possible improvement, listed as an issue at this time, relates to the precision when calculating the slope between data points when this is close to zero. Currently the value is approximated to zero when the two yValues are close together, and then the slope is calculated normally, based on xValue and yValue differences. The former could be taken into account when applying the precision as well.
      \\\\
      The only other currently significant issues relate to the exponential function classification, although it is possible that similar issues will come up with rational and other functions if they are not addressed. There are two such issues.
      \\\\
      The less significant one marks the last ten data points as "invalid" ($derivDepth=-5$) in otherwise correctly interpreted exponential data. This has to do with the end-tail  of zero derivatives that is illustrated in \Cref{tbl:mtrx:endError}. At the end of the matrix the derivative cannot be calculated as there is no succeeding data point to calculate it relative to. As such functions may not be classified correctly, and the software simply assign a common derivDepth to the last 10 data points to smooth this out. However, in exponentials this smoothing does not work correctly, perhaps because after taking the logarithm and recalculating the derivative matrix the last points are marked as POI, which in the smoothing are converted to invalid points.
      \\\\
      More significantly, however, is the way that exponential data is interpreted in its "constant regime". As stated in \Cref{sec:con:nest}, exponentials come infinitely close to zero in their asymptote, and this segment is classified as a constant function. At this stage there is nothing wrong with this, as higher level analysis can easily address this issue. However, ever so often this segment remains undefined ($derivDepth=-5$). This likely relates to the fact that this segment is usually towards the left end of the data set, as this only occurs within the first ten data points (the $maxDeppthh$ of the derivative Matrix). 
      \\\\
      None of these issues appear to have a major impact on the classifier, and those relating to the exponential only affect the last and potentially first ten data points, which is a negligible quantity compared to the total number of data points that are expected in the software's application.
      
      \subfile{tables/endErrorMatrix}
      
      There are also a number of known display issues, all of which are trivial. Given the temporary nature of the current GUI this is expected.
      \begin{itemize}
        \item Selecting "Bias" and/or "Noise" without any elected functions leads to an Exception that is handled by JavaFX. The application continues as normal, but no data is generated in response to the buggy request.
        \item The classification graph shows unnecessary lines and at times draws them acrosss the graph. This is entirely due to the way LineCharts are handled by JavaFX and cannot be addressed.
        \item Spamming the "Execute" Button quickly enough while several or many data points are being generated leads to an unhandled JavaFX exception that breaks the application.
      \end{itemize}
      
    \subsection{Immediate Next Steps}
      
      Now that the concept appears to be solid and rather well known its implementation should be completed first. At least rational functions still have to be added to the analysis, which in turn would contain a number of different possible functions. Root and logarithmic functions would follow close behind. All of these are likely to encounter similar issues to the exponential functions.
      \\\\
      Subsequently, trigonometric data could be classified by making use of an \textit{FFT} library to find periodic sections within the data. 
        
  \section{Conceptual Recommendations}
    
    The obvious next step is to actually extract the function definitions from the data i.e. to generate an equation. This will have to make use of various types of regression to extract the optimal parameters. In addition, with a somewhat well defined function, it is possible to compare adjacent segments of different function types, and for example associate the asymptode of an exponential with that exponential, rather than regarding it as an entirely separate function.
    \\\\
    The ultimate intention would be two fold. On the one hand this will provide an analytical representation of the physical phenomena in the data. This may have potential applications to model various processes. However, more importantly, its automation would allow more meaningful analysis. To discuss an example, if the expected equations describing a specific Oil Well are known detailed enough, any discrepancies will be evidence of changes, potentially to the structure or reliability of the well.
    \\\\
    Currently it appears that relatively good equations to describe the data sets serve as an abstraction to inform Machine Learning functionality, which then could gradually take on the role of the human.
    % maybe it is recommendable to have fewer but more powerful and meaningful inputs for machine learning.
    
\end{document}