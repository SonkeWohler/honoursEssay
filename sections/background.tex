\documentclass[main.tex]{subfiles}
% BACKGROUND
%
%\newglossaryentry{poly}{name={Polynomial},description={An Algebraic Function of the form $\sum_{i=0}^{N} a_i x^i$ where $N$ is the degree of the polynomial, $a_i \in \Re$ is the i^{th} coefficient and $x$ is the independent variable.}}

% rename to Background and Project
\begin{document}
  % building on introduction
  % place within bigger picture of Analysis Project
  % clarify its role within the project 
    % Data Format
    % Trace by Trace Analysis (TTA)
    % Synthetic Data?
    
  \section{The intended Application}
    
    O\&G project
    \[Depth-Temp-Data\]
    
  \section{Intended Classifications}
    
    The main objective of the project is to provide a classifier that is capable of identifying the type of function that best describes the Data Set that is analysed. This may be more than one function in different segments of the Data Set. Once a function is identified it is conceptually trivial to fit an exact equasion to the data with the use of regression \cite{}. 
    
    \subsection{Points of Interest}  
      
      At the core of this project is the identification of \textit{Points of Interest} or POI in Data Sets. A \textit{POI} is the data point at which a sudden change within the Data Set is detected. "Sudden" in the sense that it is not in accordance with the underlying pattern. For example, outliers are data points that were affected by some form of noise, like an insect passing by a visual sensor or a slammed door off-setting a seismometer sensor. As POI can describe impurities in the data they are intended to prompt higher level analysis. They may mark outliers to be simply removed, or a bias in the data that should be corrected for. 
      \\\\
      The most significant role of POI, however, is to point out interesting features in the data, that might reveal further information. To continue the seismometer example from above, certain rumblings in the ground may hint at far away or future earthquakes, and would prompt according warnings. These rumblings would break the silence that underlies the data recorded by this sensor. POI are intended to mark these phenomena for further analysis, either by a human or by more complex computational methods.
      
    \subsection{Functions}
      
      As this project aims to make use of the derivative to classify data, only functions with characteristic derivatives can be classified. The functions listed here have different characteristics that should allow some degree of certainty when classifying them.
      \paragraph{Polynomials} are of the form $f_{(x)}=\sum_{i=0}^{N} a_i x^i$ such that $i \in \mathbb{N}^\geq$, and represent perhaps the simplest form of functions that should be classified. They are commonly used in the modelling of drag, kinetics, electric current and material properties under stress or temperature changes, among many others. 
      \\\\
      In addition to this, a first order polynomial ($a x + b$) is a common component of many other functions, especially when considering $a=1$ and $b=0$, and as such may have considerable influence on these functions as well.
      \paragraph{Exponential} functions are any of the form $f_{(x)}= a A^{g(x)} + b$, where $g_{(x)}$ is a first order polynomial. Any such function is equivalent to $ f{(x)} =  a exp[  Ln(A) g_{(x)}  ] +b $ and should be modelled as such. 
      \\\\
      Exponential functions are common in the physical world, with applications in population growth, radioactive decay, quantum mechanics and more. Most significantly, it is mathematically related to many other functions.
      \paragraph{Rational} functions represent a ratio between polynomials, such that $f_{(x)}=\frac{g_{(x)}}{h_{(x)}}$. In its simplest form this is simply $f_{(x)}=\frac{1}{h_{(x)}}$, also written as $f_{(x)}=h_{(x)}^{-1}$. This is common in the strength of fields and other functions that are based on the distance from a point, due to the \textit{Inverse Square Law} \cite{}. 
      \paragraph{Roots} are an adaptation of polynomials where $i \in \mathbb{Q}^\geq$ is allowed, with the simplest example being $f_{(x)}=\sqrt{x}$. This characteristic may be of use when attempting to classify them. They are less commonly used and of lower priority.
      \paragraph{Allometrics} are functions of the form $f_{(x)}=\sum_{i=0}^{N} a_i x^{b_i} $ where $a_i,b_i \in \mathbb{R}$. They encompass polynomials, roots and rationals. Outside of these subtypes they are quite rare in the physical universe and are of very low priority.
    
    \subsection{What Should Not be Classified}
      
      As stated previously, not all functions may be classified correctly solely based on their derivative. In particular, however, periodic and other trigonometric functions can very easily be classified using an FFT \cite{}.
      \paragraph{Periodic} functions are any function that repeats itself in regular intervals, with the simplest continuous example being the \textit{sine} function of the form $a \sin(b x + c) + d$. Any periodic shape can be approximated to an arbitrary precision with a combination of \textit{sine curves}, even if the shape is discontinuous, like the \textit{square wave}. Hence, these functions are best analysed using the \textit{Fourier Transform} instead \cite{}.
      \\\\
      They represent many physical phenomena, as both \textit{oscillations} and \textit{waves} in turn represent many diverse phenomena. These include pendulae, quantum mechanics, fluid dynamics, elastic materials, certain circuits and many more.
      \paragraph{Other Trigonometrics} that are not necessarily periodic include the \textit{hyperbolic function}. As their name suggests, they are a form of \textit{trigonometric function} and are best analysed similarly, making use of \textit{FFT}, despite not being periodic. In a sense they are periodic with an infinite period. 
    
    \subsection{Combinations of Functions}
      Combinations of functions cannot always be extracted correctly. Functions can be combined for example by nesting of the form $f(g_{(x)})$, such that each function in turn affects the derivative in different ways. With some limitations, it may be possible to remove $f$ by applying its inverse, for example by taking the \textit{natural logarithm} such that $g_{(x)}=Ln\left [ f(g_{(x)})\right ]$ if $f(g_{(x)}) = e^{g(x)}$. However, unless there are reasons to suspect this particular function then all possible inverses must be applied to find the correct one. If the function is nested more than one level the possibilities grow combinatorially. 
      
      % Also show, perhaps with proof, that this would not be any better with machine learning, as the information is basically hidden until it is unppacked.
      
  \section{The Necessity of Machine Learning}
  
    As is typically the case, there are many methods available to tackle the problem at hand. Machine Learning appears to be a particularly common suggestion that is used in a true myriad of applications \cite{}. Here I wish to give an overview why, in my opinion, machine learning is an unnecessarily complex solution to this somewhat simple problem, and how it will most certainly be applicable in later project stages, making use of information extracted by the methods presented here.
    
    \subsection{The Complexity}
    
      Talk about the unpredictability of learning. By definition ML allows the resulting software to be beyond what the human can predict. Justify this with references and distil into one thought.
    
    \subsection{The Alternative}
    
      This is basically simple physics applied to and integrated in software. It can be based on rigorous mathematics and hence reduce complexity and testing.
      \\\\
      This project is intended to explore an alternative to machine learning approaches, which perhaps could be more or less efficient at the tasks at hand. 
    
    \subsubsection*{Comparable Applications}
    
      \paragraph{Edge Detection} %\hspace{0pt} \\
      also makes use of derivative \cite{}. Minor differences maybe? \\
      Very similar example of purely mathematical analysis that then informs higher level and ML analysis.
    
    \subsection{The Future}
    
      Have ML make use of this information and classify on a higher level. This way it can make use of rigorous mathematics, but fill out the gaps that are normally filled by human expertise. Additionally this approach reduces the need to teach the machine, and hence eases the way for good ML \cite{}.
    
\end{document}
