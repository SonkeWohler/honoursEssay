\documentclass[main.tex]{subfiles}
% BACKGROUND
%
%\newglossaryentry{poly}{name={Polynomial},description={An Algebraic Function of the form $\sum_{i=0}^{N} a_i x^i$ where $N$ is the degree of the polynomial, $a_i \in \Re$ is the i^{th} coefficient and $x$ is the independent variable.}}

% rename to Background and Project
\begin{document}
    
  It is important, to understand the significance of this project, to know that it is only the first phase in a larger project. The larger project should, in short, be able to identify different physical phenomena that occur in Oil and Gas wells, based on the data available from these wells. Correct and timely identification has the potential to pre-empt structural damage that would impede production, as well as shed light on yet illusive processes that, at this time, very little data is available for.
  \\\\
  \Cref{fig:well} shows a temperature-depth curve from an Oil well. This is the type of data that should be analysed. However, of each well there are several of these curves taken per day and the larger project should ultimately be able to analyse the two dimensional data sets that are created from these. 
  
  \begin{figure}[h]
    \centering
    \includegraphics[width=\linewidth]{figures/wellData}
    \caption{Temperature data from an Oil-Well in the North-Sea. Temperature is taken in approximately one meter intervals along the piping of the well, beginning on the oil rig, with 2860 data points shown. \\
    A number of phenomena can be identified on the graph, but most importantly the entry point of the well into the sea floor at about 200m, a bias at about 2200m and a number of outliers to the right of 2500m. These phenomena can be characterised as discussed in \cref{sec:POI}.}
    \caption*{Data by curtecy of HyperDap.}
    \label{fig:well}
  \end{figure}
  
  \section{Objectives}
  
    Based on examples of the available data and the appearance of these phenomena in the data, it was pointed out by a number of future project participants, that in most of these phenomena the derivative of the data may largely provide all the necessary information to at least pinpoint these phenomena and mark them for further analysis. Given the simplicity of numerical derivatives it was decided that this would be explored first.
    \\\\ 
    This project is to provide an assessment of the viability of a derivative based analysis of the available data, and create a prototype software that can serve as the foundation for further analysis. As such the following functional Requirements were agreed:
    
    \begin{enumerate}
      \item \label{it:requ:classify} (Must) Create software that can classify data sets into mathematical types.
      \item \label{it:requ:recAnalysis} (Must) Provide recommendations to expand and build on this analysis.
      \item \label{it:requ:recQual} (Should) Provide recommendations to improve the quality of classification.
      \item \label{it:requ:data} (Should) Create a prototype framework for the data to be represented internally, independently from the customer's data representation.
      \item  \label{it:requ:interp} (Could) Begin to build on data classification to extract further information from the data.
    \end{enumerate}
    
    While Requirement \ref{it:requ:recQual} mostly refers to the software's ability to deal with noise, it also hints at possible processing and memory improvements.
    \\\\
    Particularly Requirement \ref{it:requ:interp} remained intentionally vague, as at the time the nature of this further analysis was unknown (hence Requ. \ref{it:requ:recAnalysis}.
    
    Non-functional requirements were sightly more specific, to the extend to which the larger project was already defined.
    
    \begin{enumerate}
      \item \label{it:requ:java} The software must be in Java11 if possible, and usable from a Java 11 environment if that is not possible.
      \item \label{it:requ:calculus} The method of analysis should make use of calculus and focus on low level computation. Metaphorically it should provide the building blocks for further analysis.
      \item \label{it:requ:license} The software must be free of licence fees or other costs due to intellectual property.
      \item \label{it:requ:deppend} Dependencies should be kept to a minimum whenever possible.
      \item \label{it:requ:softDevel} Best software engineering practice should be followed to allow the software to be further developed without spending time on preventable quality improvement.        
    \end{enumerate}
    
    Requirements \ref{it:requ:java}, \ref{it:requ:license} and \ref{it:requ:deppend} in combination meant that, for the most part, very few  libraries could be taken advantage of. Much of the mathematics, in particular that will be recommended for further analysis, can be outsourced to various libraries, but most of these are not available for Java, and those that are have been lagging behind in updating to 11. 
    \\\\
    While Requirement \ref{it:requ:softDevel} may appear rather redundant, given the exploratory nature of the project it simply means that not all considerations should be focused on experimentation, but good quality software is ultimately the goal, which makes use of the methods that I am experimenting with.
    
  \section{Project Timeline}
    
\end{document}
