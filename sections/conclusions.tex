\documentclass[main.tex]{subfiles}
% CONCLUSION
\begin{document}
  
  In short, the project has achieved its basic objectives. It provides a prototype implementation that can be build upon and which demonstrates the viability and some of the limitations of the concepts at work.  
  
  \section{Benefits to Future Work}
    
    The future, larger project with \textit{HyperDap} will benefit from some initial development already being completed. It is provided with a family of future-proofed data types that will prove useful, and the first analysis step is ready to be build on to continue development. The next steps will largely be trivial and follow from the understanding gained in this project. 
    \\\\
    Perhaps even the presentation aspects of the software can be reused to test run and experiment with new analysis concepts. As specified by best practice it is rather trivial to expand on these modules to cover increasingly complex phenomena. 
    
  \section{Shortcomings and Limitations}
    
    Unfortunately, not all expectations were lived up to in this project. It is a rather minor shortcoming that not all functions that should be classified have been implemented. After all, the objective was a prototype and their implementation is trivial. However, it is only when assessing the software on real world examples that it can be validated to a a significant degree of certainty. In order to do this the software has to provide means of inputting large data sets, likely via text or cls files, which has not been implemented. At this stage validation can only be conducted on the demonstration functionality. While this is extremely helpful for debugging, it is not a real world example.
    
\end{document}
