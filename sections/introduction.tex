\documentclass[main.tex]{subfiles}
% INTRODUCTION
\begin{document}
  % motivation
  % intentions
  
  Calculus is a powerful tool in our ability to model, analyse and even predict the behaviour of physical phenomena all around us. It is rather disambiguous with any form of Mathematical Modelling and it can be said that, without it, technology would not be nearly as advanced as it is today.
  \\\\
  While it is figuratively the bread and butter of Scientific work, it si much less common in applied computing, to the point that many software engineers and computing students have no real understanding of it. This means that there are many problems that could be tackled much easier, but leave programmers rather helpless. Many of these applications are then solved by Machine Learning methods, some of which are essentially an adaptation of calculus and statistical mathematics. 

  \section{Objective}
    
    In this project I attempt to provide some level of classification and analysis of numerical data, specifically without the use of Machine Learning. This is on the one hand to simply to demonstrate that Machine Learning is not always rewired when one possesses a sound understanding of different mathematical methods, but on the other hand, indeed, to provide abstractions that can then be used by Machine Learning implementations as input to extract even higher level analysis.
    \\\\
    The project is to serve as a foundation to a later, larger project, to automatically detect anomalies in datasets. These datasets include various quantities and are, by-enlarge, below five dimensions. They are also very detailed representations of physical phenomena, many of which are principally well defined conceptually. As such it makes sense to use the same tools as the scientists conceptualising these phenomena. 
    
\end{document}
