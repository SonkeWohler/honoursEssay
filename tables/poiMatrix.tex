\documentclass[main.tex]{subfiles}
% POI
\begin{document}
  
  \begin{table}[h]
    \caption{Numerical example of POI classification within the derivative matrix. The data set begins with a linear pattern, and the first segment has $derivDepth=1$. Then the data changes to be 5.0 constant, triggering a POI-tail that in turn highlights the POI at index 5, which is marked as a PPOI with $derivDepth=-1$. Subsequent data points are assigned $derivDepth=0$, denoting them as constants.}
    \centering
    \begin{tabularx}{0.85\textwidth}{ X | c c c c c c c c c c }
      index & 0 & 1 & 2 & 3 & 4 & 5 & 6 & 7 & 8 & 9 \\
      xValue & 0.0 & 1.0 & 2.0 & 3.0 & 4.0 & 5.0 & 6.0 & 7.0 & 8.0 & 9.0 \\
      \hline \hline
      yValue & 0.0 & 1.0 & 2.0 & 3.0 & 4.0 & 5.0 & 5.0 &  5.0 & 5.0 & 5.0 \\
      \hline
      deriv. 1 & 1.0 & 1.0 & 1.0 & 1.0 & 1.0 & \textbf{0.0} & 0.0 & 0.0 & 0.0 & 0.0 \\
      deriv 2 & 0.0 & 0.0 & 0.0 & 0.0 & \textbf{-1.0} & 0.0 & 0.0 & 0.0 & 0.0 & 0.0 \\
      deriv 3 & 0.0 & 0.0 & 0.0 & \textbf{-1.0} & \textbf{1.0} & 0.0 & 0.0 & 0.0 & 0.0 & 0.0 \\   
      deriv 4 & 0.0 & 0.0 & \textbf{-1.0} & \textbf{2.0} & \textbf{-1.0} & 0.0 & 0.0 & 0.0 & 0.0 & 0.0 \\  
      deriv 5 & 0.0 & \textbf{-1.0} & \textbf{3.0} & \textbf{-3.0} & \textbf{1.0} & 0.0 & 0.0 & 0.0 & 0.0 & 0.0\\
      \hline
      derivDepth & 1 & 1 & 1 & 1 & 1 & \textbf{-1} & 0 & 0 & 0 & 0 \\   
    \end{tabularx}
    \label{tbl:mtrx:poi}
  \end{table}

\end{document}